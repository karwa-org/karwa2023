\problemname{\problemyamlname}


\begin{wrapfigure}{r}{5.5cm}
	\centering
	\includegraphics[width=5.5cm]{allumettes.jpg}
\end{wrapfigure}
Brieuc et Aymeric se rendent au KARWa, mais ils s'ennuient dans le train. Pour passer le temps, ils jouent aux allumettes. Le jeu consiste en $n$ allumettes,
et à chaque tour, chaque joueur peut enlever de 1 à $k$ allumettes. \textbf{Le joueur qui enlève la dernière allumette a gagné}.

Brieuc est un petit malin, et il laisse Aymeric commencer. Pouvez-vous déterminer qui va gagner la partie si les deux joueurs jouent de manière optimale ?

\begin{Input}
	Deux entiers $n$ ($2 \le n \le 10^9$) et $k$ ($1 \le k < n$), où $n$ et $k$, représentant respectivement le nombre d'allumettes et le nombre d'allumettes que chaque joueur peut enlever à chaque tour.
\end{Input}

\begin{Output}
	Le nom du gagnant du jeu si les deux joueurs jouent de manière optimale : ``\verb|Aymeric|'' si c'est le premier joueur qui gagne, ``\verb|Brieuc|'' sinon.
\end{Output}
