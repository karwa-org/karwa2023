\begin{frame}
    \frametitle{\problemtitle}
    \begin{itemize}
        \item<+-> \textbf{Problème:} Déterminer s'il existe un cycle eulérien dans un graphe non-orienté.
        \item<+-> Appliquer le théorème d'Euler: un graphe connexe admet un cycle eulérien ssi tous ses sommets ont un degré pair.
        \item<+-> On vérifie que le graphe est connexe (attention à la gare de départ qui doit aussi être considérée).
        \item<+-> On vérifie ensuite le nombre d'arêtes sortantes (c-à-d le degré) de chaque sommet.
    \end{itemize}
    % \solvestats
\end{frame}
