\problemname{\problemyamlname}

%\illustration{0.3}{image.jpg}{Caption of the illustration (optional). CC BY-NC 2.0 by X on Y}
% Source: URL to image.

% optionally define variables/limits for this problem
Dimanche matin, à 2h30, les équipes de Virgil et d'Aymeric sont en train de rentrer d'un karaoké dans la ville d'Endoven.
Malheureusement, à cette heure-ci, certaines routes sont bloquées et ils ne savent pas s'ils pourront atteindre l'arrêt de bus. 
La grille qui représente la ville est de taille N x M, avec N et M compris entre 2 et 1000.
Chaque case de la grille peut contenir un "#" pour un mur, "K" pour les deux équipes, "B" pour l'arrêt de bus et "." pour un emplacement libre. 
Pouvez-vous déterminer s'il existe un chemin pour atteindre l'arrêt de bus depuis la position des équipes ?

\begin{Input}
    Deux entiers N et M (2 <= N, M <= 2000), la taille de la grille NxM.
    N lignes de M caractères, représentant la grille.
\end{Input}

\begin{Output}
    "yes" s'il existe un chemin permettant d'atteindre l'arrêt de bus depuis les positions des équipes.
    "no" sinon.
\end{Output}
