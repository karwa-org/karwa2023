\problemname{Un horizon complexe}

La grande ville de Karwa-sur-mer souhaite recouvrir les murs lattéraux et les toits des batiments en face de la mer avec un tout nouveau type de panneau solaire pour alimenter en énergie la ville sans devoir gacher les jolies facades des dits batiments. 

Pour cela vous êtes engagé afin de déterminer combien de mètres de panneaux la ville aura besoin pour équiper ses batiments. Vous allez donc sur la plage afin de dessiner la ligne d'horizon des buildings en questions. Il ne vous reste plus qu'à déterminer combien de panneaux vous aurez besoin.


\section*{Inputs}
Vous recevrez 2 lignes :
La première contient un entier $n$ tel que $1 \leq n \leq 10 000$, la longueur en metre de la digue.
La seconde ligne représente la ligne d'horizon composé de $n$ entiers $h_{i}$ séparé d'un espace représentant la hauteur du batiment en position $i$. Notez que $0 \leq h_{i} \leq 100$

\section*{Output}
Vous devez afficher un seul entier représentant le nombre de mètre de panneaux nécessaire afin de recouvrir la totalité des côtés et dessus des batiments.