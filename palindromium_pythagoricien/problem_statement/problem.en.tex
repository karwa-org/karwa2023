\problemname{\problemyamlname}

Un débat a éclaté entre les étudiants de l'UMons et de l'UCLouvain.
Ils ne parviennent pas à se mettre d'accord sur la meilleur des deux universités, surtout dans le domaine scientifique.
Nous avons besoin de votre aide pour les départager et ainsi avantager votre université au sujet d'un tout nouveau métal appelé ``palindronium''.

Votre mission est d'optimiser la production de palindronium dans votre université, production pour l'instant très coûteuse.
Pour l'optimiser, le métal doit être fabriqué dans un moule d'une forme d'un triangle rectangle dont la longueur de l'hypothénuse est un nombre entier palindrome.
Par souci d'espace, vous ne pouvez modéliser qu'un seul moule, et celui-ci doit pouvoir rentrer dans un carré de taille $n \times n$, il doit ainsi avoir une aire maximale.

Un nombre entier palindrome est un nombre qui ne change pas s'il est lu de gauche à droite ou de droite à gauche, chiffre par chiffre.
Par exemple, $191$ est un nombre palindrome, mais $155$ ne l'est pas.
Ainsi, si le carré dans lequel votre moule doit rentrer a une taille de $15 \times 15$, vous pouvez fabriquer un moule de taille $(3,4,5)$ qui forme bien un triangle rectangle dont la longueur de l'hypothésuse, $5$, est un palindrome.
L'aire de ce moule sera $6$.
Votre but est d'avoir la plus grande aire possible sous ces contraintes !

\begin{Input}
	Un entier $n$ ($1 \le n \le 10^8$) donnant la longueur des côtés du carré devant contenir le moule triangulaire.
\end{Input}

\begin{Output}
	Une ligne contenant l'aire maximale que vous pouvez obtenir avec un moule, à une précision de $10^{-2}$.
\end{Output}
